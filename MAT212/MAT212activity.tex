\documentclass{ximera}
      
%  \input{../preamble.tex}

      
\title{MAT 212 (Calculus II) Activity}
      
      
\begin{document}
      
\begin{abstract}
      
Some sample questions that can be related to MAT 212, Calculus II.
.
      
\end{abstract}
      
\maketitle
      
      
\begin{question}      
What is $\displaystyle\int \frac{x^2+1}{x^3-2x^2+x}\ dx?$

\end{question}
\begin{explanation}
We note that the function in question is a rational function, and looking at it, does not admit a natural $u$-substitution solution.  However, like all rational functions, it does admit a partial fraction decomposition.  So notice:
\begin{eqnarray*}
\frac{x^2+1}{x^3-2x^2+x}=\frac{x^2+1}{x(x-1)^2}&=&\frac{A}{x}+\frac{B}{x-1}+\frac{C}{(x-1)^2}\\
(x(x-1)^2)\frac{x^2+1}{x(x-1)^2}&=&\left(\frac{A}{x}+\frac{B}{x-1}+\frac{C}{(x-1)^2}\right)(x(x-1)^2)\\
x^2+1&=&\answer{(x-1)^2}A+\answer{x(x-1)}B+\answer{x}C.
\end{eqnarray*}

We can at this point attempt to discern the values of $A, B, C$ by evaluating both sides of this equation for the appropriate values for $x$.  Recall, this is \textbf{not} an algebraic equality where we solve for $x$.  This is a n established equality of functions.  Thus, both sides of the equation admit the same value for every possible $x$, and we may exploit this to find our coefficients.

By evaluating $x=0$ on both sides, we obtain

\begin{eqnarray*}
\answer{1}&=&\answer{1}A+\answer{0}B+\answer{0}C\\
A&=&\answer{1}.
\end{eqnarray*}

Similarly, by evaluating $x=1$ on both sides, we obtain

\begin{eqnarray*}
\answer{2}&=&\answer{0}A+\answer{0}B+\answer{1}C\\
C&=&\answer{2}.
\end{eqnarray*}

Unfortunately, there are no other roots of our linear factors, and we have not yet obtained $B$.  However, by evaluating this equality for any other value $x$ and using the values for $A$ and $C$, we can obtain a linear equation for $B$.  Let's try $x=2$:

\begin{eqnarray*}
\answer{5}&=&\answer{1}A+\answer{2}B+\answer{2}C,\\
&& \text{then by plugging in $A, C$:}\\
\answer{5}&=&\answer{1}+\answer{2}B+\answer{4}\\
B&=&\answer{0}
\end{eqnarray*}

\textbf{ALTERNATIVELY} We recall from above that:

\begin{eqnarray*}
x^2+1&=&\answer{(x-1)^2}A+\answer{x(x-1)}B+\answer{x}C\\
&&\text{By expanding and refactoring, we obtain:}\\
x^2+1&=&(\answer{1}A+\answer{1}B+\answer{0}C)x^2\\
&&+(\answer{-2}a+\answer{-1}B+\answer{1}C)x\\
&&+(\answer{1}A+\answer{0}B+\answer{0}C)
\end{eqnarray*}

Recall that we observed that this was an equality of functions.  Specifically it's an equality of quadratic functions.  Any pair of polynomials, including quadratics, are equal if and only if their coefficients are equal.  Thus, we are able to obtain the following system of equations:

\begin{eqnarray*}
\answer{1}A+\answer{1}B+\answer{0}C&=&1,\\ \text{(The coefficients of $x^2$)}&&\\
\answer{-2}A+\answer{-1}B+\answer{1}C&=&0,\\ \text{(The coefficients of $x$)}&&\\
\answer{1}A+\answer{0}B+\answer{0}C&=&1,\\  \text{(The coefficients of $1$)}&&\\
\end{eqnarray*}
Which has solution: $A=\answer{1}, B=\answer{0}, C=\answer{2}$.

Either way you go, this gives us the decomposition:

\begin{eqnarray*}
&&\frac{x^2+1}{x^3-2x^2+x}\\
&=& \frac{A}{x}+\frac{B}{x-1}+\frac{C}{(x-1)^2}\\
&=& \frac{\answer{1}}{x}+\frac{\answer{0}}{x-1}+\frac{\answer{2}}{(x-1)^2}\\
&&\text{Then we integrate both sides:}\\
&&\int \frac{x^2+1}{x^3-2x^2+x}\ dx \\
&=& \int \frac{\answer{1}}{x}+\frac{\answer{0}}{x-1}+\frac{\answer{2}}{(x-1)^2}\ dx\\
&&\text{By the linearity of integration:}\\
&=& \int \frac{\answer{1}}{x}\ dx \\ &&+\int \frac{\answer{0}}{x-1}\ dx\\ &&+\int\frac{\answer{2}}{(x-1)^2}\ dx\\
&=& \answer{\ln|x|}+\answer{0}+\answer{\frac{-2}{x-2}}+C
\end{eqnarray*}



\end{explanation}


\begin{question}
Consider the region bounded by $y=\sqrt{x}$ and $y=x/2$.
\begin{enumerate}
\item Sketch the region bounded by these curves.
\begin{explanation}
This should be a simple review exercise, and certainly wouldn't require the use of technology.  However, it is a good opportunity to get familiar with available tools.  Use the Desmos box below and sketch the two curves.

\begin{onlineOnly}
$$\graph[xmin=0,xmax=1, ymin=0, ymax=1]{y=x/2,y=x^{.5}}$$
\end{onlineOnly}

By either observing the graph or through algebra, we notice that these curves intersect at (left most point) $(\answer{0}, \answer{0})$ and (right most point) $(\answer{4}, \answer{2})$

\end{explanation}

\item Use the washer method to find the volume of the solid of revolution generated by rotating the above region about the $x$-axis.

\begin{explanation}
Using the washer method, we note that the $x$ values in question will range from $\answer{0}$ to $\answer{4}$.  We also know that the outer radius of our washers will be $y_1=\answer{\sqrt{x}}$ and the inner radius will be $y_2=\answer{x/2}$.  So each ``slice" will have the area of a circle with the outer radius, subtracting the area of the interior circle.  Thus the volume is:

\begin{eqnarray*}
&&\int_a^b \pi(y_1^2-y_2^2)\ dx\\
&=&\int_{\answer{0}}^{\answer{4}} \pi\left(   \left( \answer{\sqrt{x}}\right)^2  - \left( \answer{x/2}\right)^2   \right)\ dx\\
&=&\int_{\answer{0}}^{\answer{4}} \pi(\answer{x-x^2/4})\ dx\\
&=&\answer{\pi(x^2/2-x^3/12)}\big|_{\answer{0}}^{\answer{4}}\\
&=&\answer{8\pi/3}.
\end{eqnarray*}



\end{explanation}




I

\item Use the cylinder method find the volume of the solid of revolution generated by rotating the above region about the $y$-axis.

\begin{explanation}
As our cylinders expand, our radii begin at $x=\answer{0}$ and stop when $y=\answer{4}$.  The height of each cylinder is the ``upper" function $y_1=\answer{\sqrt{x}}$ minus the ``lower" function $y_2=\answer{x/2}$.  Each circular ``slice" will have area equal to the height of the cylindrical slice, times the circumference of the slice.  Thus the volume is:

\begin{eqnarray*}
&&\int_a^b 2\pi x(y_1-y_x)\ dy\\
&=&\int_{\answer{0}}^{\answer{4}}2\pi x(\answer{\sqrt{x}}-\answer{x/2})\ dx\\
&=&\int_{\answer{0}}^{\answer{4}}\pi (\answer{2x^{3/2}-x^2})\ dx\\
&=&\answer{\pi(4x^{5/2}/5-x^3/3)}\big|_{\answer{0}}^{\answer{4}}\\
&=&\answer{64\pi/15}
\end{eqnarray*}


\end{explanation}




\end{enumerate}

\end{question}



\begin{question}
As $n\to \infty$ what is the behavior of the $n$th partial sum of the Taylor series for $\sin(x)$?

\begin{explanation}
This ``question" really is a thinly veiled excuse to link to another Desmos graph:

\begin{onlineOnly}
$$\graph[ymin=-2,ymax=2, panel]{y=\sin(x)}$$
\end{onlineOnly}
( The above graph is a graph of $y=\sin(x)$.  Click on \url{https://www.desmos.com/calculator/rv7resdzuu} to see a  version of this graph which displays the partial sums of the Taylor expansion.  Use the slider to control the degree of the expansion.)


We learn that as we take Taylor polynomials of larger and larger degree, that the behavior of the polynomials more closely resembles the behavior of the underlying function, particularly near the center of the expansion.  In the above Desmos graph, the slider $n$ controls the degree of the Taylor expansion.  As we increase $n$, we can see in real-time the change of behavior of the Taylor polynomials and how quickly, they begin to converge to $\sin(x)$ particularly about $x=0$.

How cool is that?


\begin{multipleChoice}
\choice[correct]{That is extremely cool!}
\choice{I don't like cool things.}
\end{multipleChoice}




\end{explanation}
\end{question}










\end{document}
