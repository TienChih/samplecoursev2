\documentclass{ximera}
%       \input{../preamble.tex}

\title{MAT 221 (Basic Concepts of Math or Math for Elementary Teachers) Activity}
      
      
\begin{document}
      
\begin{abstract}
      
Some sample questions that can be related to MAT 221, Basic Concepts of Math or Math for Elementary Teachers.
.
      
\end{abstract}
      
\maketitle
      
      
\begin{question}
Consider the set $S=\{a,b,c,\{a,b\}, d, \{d\}, \{a,b,\{e\}\}, \{\emptyset\}\}$.  

\begin{enumerate}
\item Which of the following are true?

  \begin{selectAll}
  \choice[correct]{$a\in S$}
  \choice{$a\subseteq S$}
  \choice{$\emptyset\in S$}
  \choice[correct]{$\emptyset\subseteq S$}
  \choice[correct]{$\{a,b\}\in S$}
  \choice[correct]{$\{a,b\}\subseteq S$}
    \choice[correct]{$\{a,b, d\}\subseteq S$}
      \choice{$\{a,b, e\}\subseteq S$}
  \choice{$\{\{d\}\}\in S$}
\choice{$\{\emptyset\}\subseteq S$}

  
  
    \end{selectAll}

\item Why is $\{\{d\}\}\subseteq S$?
\begin{multipleChoice}
\choice{Because $d\in S$.}
\choice[correct]{Because $\{d\}\in S$.}
\choice{Because $\{\{d\}\}\in S$.}
\end{multipleChoice}

\end{enumerate}



\end{question}


\begin{question}
Suppose that $m$ and $n$ are integers such that $3|m$.  Why does $3|mn$?
\begin{explanation}
Since $3|m$, there is an integer $k$ such that $m=\answer{3k}$.  Thus, we may rewrite $mn$ without $m$ as $mn=\answer{3kn}=3\answer{kn}$.  Clearly $3|3\answer{kn}$, so $3|mn$.
\end{explanation}
\end{question}



\begin{question}
Given the equation $$\frac{3}{x}=\frac{2}{7}$$ solve for $x$.

We can do this two ways.  A fact is that the equation $\frac{a}{b}=\frac{c}{d}$ holds if and only if $ad=bc$ holds (assuming no values are zero, but this problem is trivial if any of the terms here are zero).  Thus:

\begin{eqnarray*}
\frac{3}{x}&=& \frac{2}{7}\\
&&\text{By the above fact, we may rewrite as:}\\
\answer{3}\cdot\answer{7}&=& \answer{2}\cdot\answer{x}\\
&&\text{To isolate the $x$,}\\
&&\text{ divide both sides by $\answer{2}$:}\\
x&=& \answer{21/2}.
\end{eqnarray*}

\textbf{ALTERNATIVELY} we note that given any equality of fractions, ($\frac{a}{b}=\frac{c}{d}$) the equality of their reciprocals must hold: $\frac{b}{a}=\frac{d}{c}$ (again, assuming that $a,b,c,d\neq0$).  Thus:

\begin{eqnarray*}
\frac{3}{x}&=& \frac{2}{7}\\
&&\text{By taking reciprocals, we obtain:}\\
\frac{\answer{x}}{\answer{3}}&=& \frac{\answer{7}}{\answer{2}}\\
&&\text{To isolate $x$, we multiply both sides by} \\
&&\text{$\answer{3}$ and obtain:}\\
x&=& \answer{21/2}.
\end{eqnarray*}


\end{question}





\end{document}
